% ===== Stakeholder Poster (Fixed pipeline width + full metrics table) =====
\documentclass[21pt, a0paper, portrait,
  margin=5mm, innermargin=10mm,
  blockverticalspace=4mm, colspace=10mm, subcolspace=5mm]{tikzposter}

\usepackage{graphicx}
\usepackage{booktabs, tabularx, array}
\usepackage{amsmath}
\usepackage{enumitem}
\usepackage{siunitx}
\usepackage{hyperref}
\usepackage{tikz}
\usetikzlibrary{arrows.meta, positioning, shapes.misc, matrix, calc}
\graphicspath{{Result-figures/}}
\sisetup{round-mode=places,round-precision=4}

\usetheme{Simple}
\usecolorstyle{Default}
% High-contrast scheme
\definecolor{ocean}{HTML}{0A4D68}
\definecolor{reef}{HTML}{2A9D8F}
\definecolor{coral}{HTML}{FF6F61}
\definecolor{sand}{HTML}{F7F3E9}
\colorlet{titlebgcolor}{ocean}
\colorlet{titlefgcolor}{white}
\colorlet{blocktitlebgcolor}{sand!60}
\colorlet{blocktitlefgcolor}{ocean}
\colorlet{backgroundcolor}{sand}

\author{Coral-MTL: Hierarchical Multi-Task Learning for Coral Reef Health Assessment\vspace{15pt}}
\institute{\textbf{Group 21} - Capstone Data Challenge (JBG060) - Eindhoven University of Technology \& ReefSupport}

\setlist[itemize]{leftmargin=1.05em,itemsep=0.22em,topsep=0.1em}
\setlist[enumerate]{leftmargin=1.3em,itemsep=0.22em,topsep=0.1em}

\makeatletter
\settitle{%
  \vbox{
    \begin{minipage}[c]{1.5\linewidth}
      \color{titlefgcolor}
      {\bfseries\scshape\VeryHuge \@title \par}
      {\huge \@author \par}
      {\Large \@institute \par}
    \end{minipage}\hfill
    \begin{minipage}[c]{0.25\linewidth}
      \raggedleft
      % \includegraphics[width=\linewidth]{qr_repo.png}
    \end{minipage}
  }
}
\makeatother

\begin{document}
\color{black}
\maketitle

\begin{columns}

% ===================== LEFT COLUMN =====================================================
\column{0.33}

\block{(A) Context \& Motivation\vspace{-30pt}}{
\textbf{Stakeholder need.} Imagery volume outpaces expert capacity. Actions require \emph{dense, boundary-accurate} maps answering \textbf{what} coral (genus) and \textbf{how} it is doing (alive/bleached/dead).

\textbf{Why dense masks.} Point tools remain semi-manual and boundary-blind; cover \& lesion tracking depend on \textbf{perimeters}. Dense segmentation outputs areas \& edges directly.
}

\block{(B) Research Objective\vspace{-30pt}}{
\textbf{Goal.} Build and evaluate a \emph{single} deployable model that jointly predicts \textbf{Genus} and \textbf{Health} masks with \textbf{better boundary quality} than a strong single-task baseline, while \textbf{reporting} probability reliability (ECE/NLL/Brier). \emph{We do not apply post-hoc calibration.}}

\block{(C) Data \& Assumptions\vspace{-30pt}}{
\textbf{Dataset.} \emph{Coralscapes dataset}; dense pixel labels over \textbf{39 benthic classes} incl.\ live/bleached/dead; in varied depth, turbidity, illumination.

\textbf{Splits.} \emph{Site-level hold-out}; near-duplicate \& temporal separation; \emph{Poisson-disk sampling (PDS)} to reduce redundancy.

\textbf{Augmentations.} Physics-plausible haze, color cast, blur.

\textbf{Assumption.} Genus (morphology) and health (appearance) are complementary and should be learned together.
}

\block{(D) Approach \& Design Rationale\vspace{-30pt}}{
\textbf{Architecture.} SegFormer encoder with \textbf{two primary heads} (Genus, Health) using \emph{explicit feature exchange}; \textbf{five lightweight auxiliaries} (Fish, Human artifacts, Substrate, Background, Biota) provide context to sharpen edges.

\textbf{Losses/Optimization.} Dice + Focal; \textbf{IMGrad} for task-gradient balancing; 20 epochs. Both MTL variants in our results use IMGrad.

\textbf{Reasoning.} Shared features + explicit exchange exploit complementarity; auxiliaries regularize and reduce background confusions without large compute cost.
}

% --------- (E) End-to-End Pipeline (fits column, no overflow) ----------
\block{(E) End-to-End Pipeline\vspace{-30pt}}{
\begin{center}
\begin{tikzpicture}[
  font=\Large,
  node distance=7mm,
  >=Latex,
  rounded corners=10pt
]
\tikzset{
  box/.style={
    draw=titlebg, very thick, fill=white, align=center,
    text width=0.90\linewidth, minimum height=1.8cm
  }
}

\node[box] (img)  {\textbf{Field Imagery}\\  Video / Photos};
\node[box, below=of img] (tile) {\textbf{Tiling \& Sampling}\\  Poisson-disk};
\node[box, below=of tile] (aug)  {\textbf{Augmentations}\\  Haze / Blur / Color};
\node[box, below=of aug]  (enc)  {\textbf{SegFormer B2 Encoder}\\  Multi-scale features};
\node[box, below=of enc]  (dual) {\textbf{Dual Heads with Feature Exchange}\\\emph{Genus} \& \emph{Health}};
\node[box, below=of dual] (aux)  {\textbf{Auxiliary Heads (optional)}\\  Fish / Human Artifacts / Substrate / Background / Biota};
\node[box, below=of aux]  (train){\textbf{Losses \& Optimization}\\  Dice + Focal + IMGrad};
\node[box, below=of train]  (eval) {\textbf{Evaluation}\\  Global mIoU / Global BIoU / Boundary-F1 / ECE / NLL / Brier};
\node[box, below=of eval] (out)  {\textbf{Outputs}\\  Masks};

\draw[->, very thick] (img)  -- (tile);
\draw[->, very thick] (tile) -- (aug);
\draw[->, very thick] (aug)  -- (enc);
\draw[->, very thick] (enc)  -- (dual);
\draw[->, very thick] (dual) -- (aux);
\draw[->, very thick] (aux)  -- (train);
\draw[->, very thick] (train) -- (eval);
\draw[->, very thick] (eval) -- (out);
\end{tikzpicture}
\end{center}
}

% ===================== MIDDLE COLUMN ===================================================
\column{0.34}

\block{(F) MTL Variants - Focused vs Holistic\vspace{-30pt}
}{
\textbf{Shared core:} SegFormer B2 encoder + two primary heads (\textbf{Genus}, \textbf{Health}) with feature exchange.

\textbf{MTL Focused (2-task).}
\begin{itemize}
  \item \emph{What:} Only genus \& health heads.
  \item \emph{Why:} Optimizes on the two target tasks.
\end{itemize}

\textbf{MTL Holistic (7-task).}
\begin{itemize}
  \item \emph{What:} Sets all tasks as primary (Genus, Health, Fish, Human artifacts, Substrate, Background, Biota).
  \item \emph{Why:} Context regularizes features and sharpens coral/non-coral boundaries.
\end{itemize}
}

\vspace{-50pt}
\block{(G) Metrics - what and why\vspace{-30pt}}{
\textbf{Global mIoU} - class-mean IoU (overall agreement). \\
\textbf{Global BIoU} - IoU on boundary bands (perimeter fidelity). \\
\textbf{Boundary F1} - boundary precision/recall (over/under-seg). \\
\textbf{Calibration (reported only)} - \textbf{ECE}/\textbf{NLL}/\textbf{Brier} describe probability reliability.

\textit{Reasoning.} Decisions rely on perimeters and trustworthy probabilities; therefore boundary metrics are first-class, and calibration is documented for transparency.}

\vspace{-50pt}
\block{(H) Headline Results (KPIs)\vspace{-30pt}}{
\includegraphics[width=\linewidth]{model_comparison_bar_3models.png}

\textbf{Test set} (percentage gains). \textbf{MTL Holistic} improves over
\textbf{Baseline} by \emph{+9.9\% mIoU}, \emph{+32.6\% Global BIoU}, and
\emph{+28.9\% Boundary-F1}; and over \textbf{MTL Focused} by
\emph{+5.8\% mIoU}, \emph{+15.6\% Global BIoU}, and \emph{+13.9\% Boundary-F1}.

\textit{Meaning.} Cleaner perimeters $\Rightarrow$ better cover estimates and more faithful lesion-change boundaries.\vspace{50pt}
{\renewcommand{\arraystretch}{1.22}\large
\begin{tabularx}{\linewidth}{@{}l *{3}{>{\centering\arraybackslash}X}@{}}
\toprule
\textbf{Metric} & \textbf{Baseline} & \textbf{MTL Focused} & \textbf{MTL Holistic} \\
\midrule
Global mIoU $\uparrow$   & 0.3888 & 0.4039 & \textbf{0.4272} \\
Global BIoU $\uparrow$   & 0.0937 & 0.1075 & \textbf{0.1243} \\
Boundary F1 $\uparrow$   & 0.1714 & 0.1942 & \textbf{0.2211} \\
ECE $\downarrow$         & \textbf{0.1014} & 0.1275 & 0.1423 \\
NLL $\downarrow$         & \textbf{1.2239} & 1.3995 & 1.5162 \\
Brier $\downarrow$       & 0.5016 & 0.4959 & \textbf{0.4937} \\
\bottomrule
\end{tabularx}

\textit{Interpretation.} Holistic is best on segmentation/boundary; baseline is slightly better calibrated (lower ECE/NLL).}}

\vspace{-50pt}
\block{(I) What changed under the hood\vspace{-30pt}}{
\includegraphics[width=\linewidth]{error_decomposition_3models.png}

\textbf{Positives.} \textbf{Fewer classification errors} and \textbf{fewer background false positives} from Baseline $\rightarrow$ Focused $\rightarrow$ \textbf{Holistic}. \\
\textbf{Negative.} \textbf{Missed regions (FN)} increase, conservative behavior in low contrast. \\
\textit{Operational meaning.} Risk of \emph{underestimating} extent under poor visibility; mitigate via targeted data \& modest class-weighted focal on faint bleaching.
}

% ===================== RIGHT COLUMN ====================================================
\column{0.33}

\block{}{
\includegraphics[width=\linewidth]{training_progress_3models.png}

\textbf{Learning dynamics.} Holistic improves earlier and plateaus higher by $\sim$20 epochs $\Rightarrow$ prefer early stop and spend time on error analysis/data curation rather than longer training.
}

\vspace{-50pt}
\block{(J) Class-Level Insights\vspace{-30pt}}{
  \includegraphics[width=\linewidth]{genus_radar_comparison_all_models.png}\\
  \textit{Genus} segmentation by model (IoU): Holistic MTL matches or exceeds others across most genera.\vspace{30pt}
  
  \includegraphics[width=\linewidth]{health_radar_comparison_all_models.png}\\
  \textit{Health} segmentation by model (IoU): Holistic MTL shows consistent gains, largest for bleached coral.”
}

\vspace{-50pt}
\block{(K) Choosing a model (evidence-based)\vspace{-30pt}}{
\textbf{Choose \underline{MTL Holistic}} when boundary quality (BIoU/Boundary-F1) drives value (cover/lesion tracking) and background FPs are costly.

\textbf{Choose \underline{Baseline}} when you prioritize raw probability reliability (lower ECE/NLL) and cannot yet curate low-contrast data to reduce FNs.

\textbf{MTL Focused} is a middle ground with moderate boundary gains and smaller change in error profile.
}

\vspace{-50pt}
\block{(L) Next Steps (specific \& testable)\vspace{-30pt}}{
\textbf{Data/aug:} curate low-contrast/whitened scenes; mine algae-on-dead vs.\ rock; add rubble exemplars; boundary-focused augmentation (random contour erode/dilate). \\
\textbf{Loss/opt:} modest class-weighted focal for faint bleaching; keep IMGrad for balance. \\
\textbf{Validation:} cross-region hold-outs on partner imagery; monitor \textbf{mIoU}, \textbf{BIoU}/\textbf{Boundary-F1}, and \textbf{ECE/NLL/Brier}.
}

\end{columns}
\end{document}
